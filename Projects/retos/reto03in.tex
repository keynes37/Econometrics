% Options for packages loaded elsewhere
\PassOptionsToPackage{unicode}{hyperref}
\PassOptionsToPackage{hyphens}{url}
\PassOptionsToPackage{dvipsnames,svgnames*,x11names*}{xcolor}
%
\documentclass[
]{article}
\usepackage{lmodern}
\usepackage{amssymb,amsmath}
\usepackage{ifxetex,ifluatex}
\ifnum 0\ifxetex 1\fi\ifluatex 1\fi=0 % if pdftex
  \usepackage[T1]{fontenc}
  \usepackage[utf8]{inputenc}
  \usepackage{textcomp} % provide euro and other symbols
\else % if luatex or xetex
  \usepackage{unicode-math}
  \defaultfontfeatures{Scale=MatchLowercase}
  \defaultfontfeatures[\rmfamily]{Ligatures=TeX,Scale=1}
  \setmainfont[]{Serif}
  \setsansfont[]{Fira Sans}
  \setmonofont[]{Fira Code}
\fi
% Use upquote if available, for straight quotes in verbatim environments
\IfFileExists{upquote.sty}{\usepackage{upquote}}{}
\IfFileExists{microtype.sty}{% use microtype if available
  \usepackage[]{microtype}
  \UseMicrotypeSet[protrusion]{basicmath} % disable protrusion for tt fonts
}{}
\makeatletter
\@ifundefined{KOMAClassName}{% if non-KOMA class
  \IfFileExists{parskip.sty}{%
    \usepackage{parskip}
  }{% else
    \setlength{\parindent}{0pt}
    \setlength{\parskip}{6pt plus 2pt minus 1pt}}
}{% if KOMA class
  \KOMAoptions{parskip=half}}
\makeatother
\usepackage{xcolor}
\IfFileExists{xurl.sty}{\usepackage{xurl}}{} % add URL line breaks if available
\IfFileExists{bookmark.sty}{\usepackage{bookmark}}{\usepackage{hyperref}}
\hypersetup{
  pdftitle={Econometría I},
  colorlinks=true,
  linkcolor=Maroon,
  filecolor=Maroon,
  citecolor=Blue,
  urlcolor=blue,
  pdfcreator={LaTeX via pandoc}}
\urlstyle{same} % disable monospaced font for URLs
\usepackage[margin=1in]{geometry}
\usepackage{longtable,booktabs}
% Correct order of tables after \paragraph or \subparagraph
\usepackage{etoolbox}
\makeatletter
\patchcmd\longtable{\par}{\if@noskipsec\mbox{}\fi\par}{}{}
\makeatother
% Allow footnotes in longtable head/foot
\IfFileExists{footnotehyper.sty}{\usepackage{footnotehyper}}{\usepackage{footnote}}
\makesavenoteenv{longtable}
\usepackage{graphicx}
\makeatletter
\def\maxwidth{\ifdim\Gin@nat@width>\linewidth\linewidth\else\Gin@nat@width\fi}
\def\maxheight{\ifdim\Gin@nat@height>\textheight\textheight\else\Gin@nat@height\fi}
\makeatother
% Scale images if necessary, so that they will not overflow the page
% margins by default, and it is still possible to overwrite the defaults
% using explicit options in \includegraphics[width, height, ...]{}
\setkeys{Gin}{width=\maxwidth,height=\maxheight,keepaspectratio}
% Set default figure placement to htbp
\makeatletter
\def\fps@figure{htbp}
\makeatother
\setlength{\emergencystretch}{3em} % prevent overfull lines
\providecommand{\tightlist}{%
  \setlength{\itemsep}{0pt}\setlength{\parskip}{0pt}}
\setcounter{secnumdepth}{-\maxdimen} % remove section numbering
%% See: https://bookdown.org/yihui/rmarkdown-cookbook/multi-column.html
%% I've made some additional adjustments based on my own preferences (e.g. cols
%% should be top-aligned in case of uneven vertical length)
\renewcommand{\contentsname}{Contenido}
\newenvironment{columns}[1][]{}{}

\newenvironment{column}[1]{\begin{minipage}[t]{#1}\ignorespaces}{%
\end{minipage}
\ifhmode\unskip\fi
\aftergroup\useignorespacesandallpars
}

\def\useignorespacesandallpars#1\ignorespaces\fi{%
#1\fi\ignorespacesandallpars}

\makeatletter
\def\ignorespacesandallpars{%
  \@ifnextchar\par
    {\expandafter\ignorespacesandallpars\@gobble}%
    {}%
}
\makeatother
\usepackage{booktabs}
\usepackage{threeparttable}
\usepackage{float}

\title{Econometría I}
\usepackage{etoolbox}
\makeatletter
\providecommand{\subtitle}[1]{% add subtitle to \maketitle
  \apptocmd{\@title}{\par {\large #1 \par}}{}{}
}
\makeatother
\subtitle{Reto 04}
\usepackage{authblk}
                                        \author[]{Carlos A. Yanes
Guerra}
                                                            \affil{Universidad
del Norte \textbar{} Departamento de Economía}
                                            \date{}

\begin{document}
\maketitle

{
\hypersetup{linkcolor=}
\setcounter{tocdepth}{3}
\tableofcontents
}
\subsection{Antes de empezar}\label{antes-de-empezar}

Recuerde que los retos son para desarrollarlos en clases en el tiempo
sugerido por el profesor 1H:20 Min. Trate con su grupo de trabajo
(máximo 2 personas) este -tenga un toque único- donde haga uso en la
gran mayoría de los códigos de clase y no los que les da la IA.

\subsubsection{Objetivo}\label{objetivo}

Hacer uso del modelo base de Machine learning (regresión)

\subsubsection{Datos a usar}\label{datos-a-usar}

Sea la siguiente tabla con información:

\begin{longtable}[]{@{}
  >{\raggedright\arraybackslash}p{(\columnwidth - 10\tabcolsep) * \real{0.0694}}
  >{\raggedright\arraybackslash}p{(\columnwidth - 10\tabcolsep) * \real{0.2361}}
  >{\raggedright\arraybackslash}p{(\columnwidth - 10\tabcolsep) * \real{0.0833}}
  >{\raggedright\arraybackslash}p{(\columnwidth - 10\tabcolsep) * \real{0.1806}}
  >{\raggedright\arraybackslash}p{(\columnwidth - 10\tabcolsep) * \real{0.1528}}
  >{\raggedright\arraybackslash}p{(\columnwidth - 10\tabcolsep) * \real{0.2778}}@{}}
\toprule\noalign{}
\begin{minipage}[b]{\linewidth}\raggedright
Obs
\end{minipage} & \begin{minipage}[b]{\linewidth}\raggedright
Nombre
\end{minipage} & \begin{minipage}[b]{\linewidth}\raggedright
Edad
\end{minipage} & \begin{minipage}[b]{\linewidth}\raggedright
Altura (cm)
\end{minipage} & \begin{minipage}[b]{\linewidth}\raggedright
Peso (kg)
\end{minipage} & \begin{minipage}[b]{\linewidth}\raggedright
Horas de Ejercicio
\end{minipage} \\
\midrule\noalign{}
\endhead
\bottomrule\noalign{}
\endlastfoot
1 & Ana Martinez & 30 & 165 & 60 & 5 \\
2 & Carlos Perez & 45 & 175 & 80 & 3 \\
3 & Laura Gomez & 25 & 160 & 55 & 6 \\
4 & Juan Rodriguez & 35 & 180 & 85 & 4 \\
5 & Maria Lopez & 28 & 170 & 65 & 5 \\
6 & Pedro Sanchez & 40 & 178 & 78 & 2 \\
7 & Lucia Fernandez & 32 & 168 & 62 & 4 \\
8 & Jose Ramirez & 50 & 182 & 90 & 1 \\
9 & Sofia Torres & 27 & 162 & 58 & 5 \\
10 & Andres Suarez & 38 & 176 & 82 & 3 \\
11 & Paula Castro & 33 & 167 & 63 & 4 \\
12 & Miguel Vargas & 29 & 174 & 75 & 3 \\
13 & Elena Morales & 26 & 161 & 57 & 6 \\
14 & Diego Rios & 42 & 179 & 83 & 2 \\
15 & Natalia Ruiz & 31 & 166 & 61 & 5 \\
\end{longtable}

\subsection{Preguntas del Reto}\label{preguntas-del-reto}

\begin{enumerate}
\def\labelenumi{\arabic{enumi}.}
\item
  Establezca primero con las variables (Altura y Peso) un
  \textbf{diagrama de dispersión}, luego haga lo mismo pero ahora con
  las variables (Altura y Horas de Ejercicio). Realice un análisis
  comparativo de ambas gráficas.
\item
  Haga un análisis del peso por edad. Pista: Puede ayudarse del comando
  \texttt{Table} de R.
\item
  Cree una variable \emph{dicotómica} o dummy para el género de la
  persona y establezca una comparación entre el peso por género de las
  personas de la base. Pista: Debe responder finalmente cuál género
  tiene mayor peso promedio.
\item
  Realice una prueba estadística para determinar si el promedio de horas
  de ejercicio es diferente por género. Pista: Haga uso de la
  \textbf{prueba T de Student}. Explique y plantee la prueba de
  hipótesis. Corra ahora una \emph{regresión con las variables}.
  ¿Difiere de lo obtenido en la prueba T de Student? Explique
\item
  Ejecute una regresión adicional, pero ahora cuando la variable
  dependiente es el peso y la explicativa es la altura. Pista: Muestre
  los resultados, incluyendo valores predichos y residuales e interprete
  el modelo.
\end{enumerate}

\end{document}